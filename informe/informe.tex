\documentclass[a4paper, 10pt, twoside]{article}

\usepackage[top=1in, bottom=1in, left=1in, right=1in]{geometry}
\usepackage[utf8]{inputenc}
\usepackage[spanish, es-ucroman, es-noquoting]{babel}
\usepackage{setspace}
\usepackage{fancyhdr}
\usepackage{lastpage}
\usepackage{amsmath}
\usepackage{amsfonts}
\usepackage{amsthm}
\usepackage{graphicx}
\usepackage{float}
\usepackage{enumitem} % Provee macro \setlist
\usepackage{tabularx}
\usepackage{multirow}
\usepackage{hyperref}
\usepackage{multicol}
\usepackage{verbatim}
\usepackage{listings}
\usepackage[toc, page]{appendix}
\usepackage{color}


%%%%%%%%%% Configuración de Fancyhdr - Inicio %%%%%%%%%%
\pagestyle{fancy}
\thispagestyle{fancy}
\lhead{Trabajo Práctico · Investigación Operativa}
\rhead{Aboy Solanes · COMPLETAR}
\renewcommand{\footrulewidth}{0.4pt}
\cfoot{\thepage /\pageref{LastPage}}

\fancypagestyle{caratula} {
   \fancyhf{}
   \cfoot{\thepage /\pageref{LastPage}}
   \renewcommand{\headrulewidth}{0pt}
   \renewcommand{\footrulewidth}{0pt}
}
%%%%%%%%%% Configuración de Fancyhdr - Fin %%%%%%%%%%


%%%%%%%%%% Miscelánea - Inicio %%%%%%%%%%
% Evita que el documento se estire verticalmente para ocupar el espacio vacío
% en cada página.
\raggedbottom

% Deshabilita sangría en la primer línea de un párrafo.
\setlength{\parindent}{0em}

% Separación entre párrafos.
\setlength{\parskip}{0.5em}

% Separación entre elementos de listas.
\setlist{itemsep=0.5em}

% Asigna la traducción de la palabra 'Appendices'.
\renewcommand{\appendixtocname}{Apéndices}
\renewcommand{\appendixpagename}{Apéndices}
%%%%%%%%%% Miscelánea - Fin %%%%%%%%%%


%%%%%%%%%% Insertar diagrama - Inicio %%%%%%%%%%
\newcommand{\diagramav}[1]{%
  \includegraphics[type=pdf,ext=.pdf,read=.pdf,width=16cm]{diagramas/#1}%
}

\newcommand{\diagramavfig}[2]{%
  \begin{figure}[H]
    \includegraphics[type=pdf,ext=.pdf,read=.pdf,width=16cm]{diagramas/#1}%
    \caption{#2}
    \label{fig:#1}
  \end{figure}
}

\newcommand{\diagramavtrim}[2]{%
  \includegraphics[type=pdf,ext=.pdf,read=.pdf,width=16cm,trim=0 #2 0 0,clip]{diagramas/#1}%
}

\newcommand{\diagramah}[1]{%
  \includegraphics[type=pdf,ext=.pdf,read=.pdf,height=16cm,angle=90]{diagramas/#1}%
}
%%%%%%%%%% Insertar diagrama - Fin %%%%%%%%%%


\begin{document}


%%%%%%%%%%%%%%%%%%%%%%%%%%%%%%%%%%%%%%%%%%%%%%%%%%%%%%%%%%%%%%%%%%%%%%%%%%%%%%%
%% Carátula                                                                  %%
%%%%%%%%%%%%%%%%%%%%%%%%%%%%%%%%%%%%%%%%%%%%%%%%%%%%%%%%%%%%%%%%%%%%%%%%%%%%%%%


\thispagestyle{caratula}

\begin{center}

\includegraphics[height=2cm]{DC.png} 
\hfill
\includegraphics[height=2cm]{UBA.jpg} 

\vspace{2cm}

Departamento de Computación,\\
Facultad de Ciencias Exactas y Naturales,\\
Universidad de Buenos Aires

\vspace{4cm}

\begin{Huge}
Trabajo Práctico 1
\end{Huge}

\vspace{0.5cm}

\begin{Large}
Investigación Operativa
\end{Large}

\vspace{1cm}

Segundo Cuatrimestre de 2015

\vspace{4cm}

\begin{tabular}{|c|c|c|}
\hline
Apellido y Nombre & LU & E-mail\\
\hline
Santiago Aboy Solanes & 175/12 & santiaboy2@hotmail.com \\
Pedro COMPLETAR & COMP/ETAR & COMPLETAR@COMPLETAR.com \\
\hline
\end{tabular}

\end{center}

\newpage

\tableofcontents

\newpage


%\section{Introducción}

%\section{Desarrollo}

%\subsection{Decisiones tomadas}

%\subsection{Problemas encontrados}

%\subsection{Conclusiones}

\[
 x_{pj} =
  \begin{cases} 
      \hfill 1    \hfill & \text{si el color $j$ es asignado al vértice $p$} \\
      \hfill 0 \hfill & \text{caso contrario} \\
  \end{cases}
\]


\[
 w_j =
  \begin{cases} 
      \hfill 1    \hfill & \text{si $x_{pj}$ = 1 para algún vértice $p$} \\
      \hfill 0 \hfill & \text{caso contrario} \\
  \end{cases}
\]

\begin{align*}
Min \sum_{j=1}^{n} w_j \qquad & s.a.:\\
\sum_{v_i \in V_p}\sum_{j=1}^{n} x_{ij} \qquad & \forall \ V_p \in \{V_1, \ldots, V_k\} \text{//un nodo coloreado por particion}\\
x_{ij} + x_{i'j} \leq 1 \qquad & \forall \ (i,i') \in X, \ \forall \ j = 1, \ldots, n \text{//nodos adyacentes no tienen el mismo color}\\
x_{ij} \leq w_j \qquad & \forall \ v_i \in V, \ \forall \ j=1,\ldots,n \text{//desigualdad color vs nodos con ese color}\\
x_{ij} \in \{0,1\}\\
w_j \in \{0,1\}\\
\end{align*}


\end{document}
