\documentclass[a4paper, 10pt, twoside]{article}

\usepackage[top=1in, bottom=1in, left=1in, right=1in]{geometry}
\usepackage[utf8]{inputenc}
\usepackage[spanish, es-ucroman, es-noquoting]{babel}
\usepackage{setspace}
\usepackage{fancyhdr}
\usepackage{lastpage}
\usepackage{amsmath}
\usepackage{amsfonts}
\usepackage{amsthm}
\usepackage{graphicx}
\usepackage{float}
\usepackage{enumitem} % Provee macro \setlist
\usepackage{tabularx}
\usepackage{multirow}
\usepackage{hyperref}
\usepackage{multicol}
\usepackage{verbatim}
\usepackage{listings}
\usepackage[toc, page]{appendix}
\usepackage{color}
\usepackage{listings}

\usepackage[lined, boxed, linesnumbered, commentsnumbered]{algorithm2e}

\lstset{language=C++} 

\definecolor{listinggray}{gray}{0.9}
\definecolor{lbcolor}{rgb}{0.9,0.9,0.9}
\lstdefinestyle{customc++}{
    backgroundcolor=\color{lbcolor},
    tabsize=4,
  language=C++,
  captionpos=b,
  tabsize=3,
  frame=lines,
  numbers=left,
  numberstyle=\tiny,
  numbersep=5pt,
  breaklines=true,
  showstringspaces=false,
  basicstyle=\footnotesize,
%  identifierstyle=\color{magenta},
  keywordstyle=\color[rgb]{0,0,1},
  %commentstyle=\color{green},
  stringstyle=\color{red}
}

\lstset{escapechar=@,style=customc++}

%%%%%%%%%% Configuración de Fancyhdr - Inicio %%%%%%%%%%
\pagestyle{fancy}
\thispagestyle{fancy}
\lhead{Trabajo Práctico · Investigación Operativa}
\rhead{Aboy Solanes · Rodriguez}
\renewcommand{\footrulewidth}{0.4pt}
\cfoot{\thepage /\pageref{LastPage}}

\fancypagestyle{caratula} {
   \fancyhf{}
   \cfoot{\thepage /\pageref{LastPage}}
   \renewcommand{\headrulewidth}{0pt}
   \renewcommand{\footrulewidth}{0pt}
}
%%%%%%%%%% Configuración de Fancyhdr - Fin %%%%%%%%%%


%%%%%%%%%% Miscelánea - Inicio %%%%%%%%%%
% Evita que el documento se estire verticalmente para ocupar el espacio vacío
% en cada página.
\raggedbottom

% Deshabilita sangría en la primer línea de un párrafo.
\setlength{\parindent}{0em}

% Separación entre párrafos.
\setlength{\parskip}{0.5em}

% Separación entre elementos de listas.
\setlist{itemsep=0.5em}

% Asigna la traducción de la palabra 'Appendices'.
\renewcommand{\appendixtocname}{Apéndices}
\renewcommand{\appendixpagename}{Apéndices}
%%%%%%%%%% Miscelánea - Fin %%%%%%%%%%


%%%%%%%%%% Insertar diagrama - Inicio %%%%%%%%%%
\newcommand{\diagramav}[1]{%
  \includegraphics[type=png,ext=.png,read=.png,width=16cm]{graficos/#1}%
}

\newcommand{\diagramavfig}[2]{%
  \begin{figure}[H]
    \includegraphics[type=png,ext=.png,read=.png,width=16cm]{graficos/#1}%
    \caption{#2}
    \label{fig:#1}
  \end{figure}
}

\newcommand{\diagramavtrim}[2]{%
  \includegraphics[type=png,ext=.png,read=.png,width=16cm,trim=0 #2 0 0,clip]{graficos/#1}%
}

\newcommand{\diagramah}[1]{%
  \includegraphics[type=png,ext=.png,read=.png,height=16cm,angle=90]{graficos/#1}%
}
%%%%%%%%%% Insertar diagrama - Fin %%%%%%%%%%


\begin{document}


%%%%%%%%%%%%%%%%%%%%%%%%%%%%%%%%%%%%%%%%%%%%%%%%%%%%%%%%%%%%%%%%%%%%%%%%%%%%%%%
%% Carátula                                                                  %%
%%%%%%%%%%%%%%%%%%%%%%%%%%%%%%%%%%%%%%%%%%%%%%%%%%%%%%%%%%%%%%%%%%%%%%%%%%%%%%%


\thispagestyle{caratula}

\begin{center}

\includegraphics[height=2cm]{DC.png} 
\hfill
\includegraphics[height=2cm]{UBA.jpg} 

\vspace{2cm}

Departamento de Computación,\\
Facultad de Ciencias Exactas y Naturales,\\
Universidad de Buenos Aires

\vspace{4cm}

\begin{Huge}
Trabajo Práctico 1
\end{Huge}

\vspace{0.5cm}

\begin{Large}
Investigación Operativa
\end{Large}

\vspace{1cm}

Segundo Cuatrimestre de 2015

\vspace{4cm}

\begin{tabular}{|c|c|c|}
\hline
Apellido y Nombre & LU & E-mail\\
\hline
Santiago Aboy Solanes & 175/12 & santiaboy2@hotmail.com \\
Pedro Rodriguez & 197/12 & pedro3110.jim@gmail.com \\
\hline
\end{tabular}

\end{center}

\newpage

\tableofcontents

\newpage


\section{Introducción}

El problema de coloreo particionado de grafos es una generalización del problema de coloreo de grafos. Al igual que en coloreo, se tiene un grafo $G = (V,E)$ siendo $V$ los nodos del grafo y $E$ sus aristas.

En coloreo particionado, el conjunto $V$ se encuentra dividido en una partición $V_1, \ldots, V_k$. El objetivo es asignar un color a sólo un vértice de cada partición de manera tal que dos vértices adyacentes no reciban el mismo color, minimizando la cantidad de colores utilizados.

\section{Modelos Utilizados}

En este trabajo práctico desarrollamos dos modelos distintos (Modelo Pedro y Modelo Santiago) basándonos en las mismas variables:

\[
 x_{pj} =
  \begin{cases} 
      \hfill 1    \hfill & \text{si el color $j$ es asignado al vértice $p$} \\
      \hfill 0 \hfill & \text{caso contrario} \\
  \end{cases}
\]


\[
 w_j =
  \begin{cases} 
      \hfill 1    \hfill & \text{si $x_{pj}$ = 1 para algún vértice $p$} \\
      \hfill 0 \hfill & \text{caso contrario} \\
  \end{cases}
\]

\subsection{Modelo Pedro}

\textit{Aclaracion:} (1) y (2) no son escalares sino que van a ser usados para referenciar a las inecuaciones.

\label{modelo-pedro}

$Min \sum_{j=1}^{n} w_j$ sujeto a:
\begin{align*}
(1)\ \sum_{v_i \in V_p}\sum_{j=1}^{n} x_{ij} = 1 \qquad & \forall \ V_p \in \{V_1, \ldots, V_k\}\\
(2)\ x_{ij} + x_{i'j} \leq 1 \qquad & \forall \ (i,i') \in E, \ \forall \ j = 1, \ldots, n, \ x_i \ \text{y} \ x_{i'} \text{en distinta partición}\\
k * w_j \geq \sum_{i=1}^{n} x_{ij} \qquad & \forall \ j \in \{1, \ldots, k\}\\
w_j \leq \sum_{i=1}^{n} x_{ij} \qquad & \forall \ j \in \{1, \ldots, k\}\\
x_{ij} \in \{0,1\}\\
w_j \in \{0,1\}\\
\end{align*}

(1) es la restricción que establece que haya un único nodo coloreado por partición.

(2) indica que los nodos adyacentes de distintas particiones no tienen el mismo color. No hace falta forzar esto para la misma partición ya que (1) se encarga de eso.

Las siguientes dos restricciones se encargan de mantener el balance entre los colores y los nodos con ese color.

\subsection{Modelo Santiago}

Este modelo es igual a Modelo Pedro, salvo que las desigualdades de color vs nodos con ese color se modelan con una sola restricción de la siguiente forma:

\begin{align*}
x_{ij} \leq w_j \ \forall \ v_i \in V, \ \forall \ j=1,\ldots,n
\end{align*}

Este último modelo tiene más restricciones que el anterior, dado un mismo grafo. En Modelo Pedro, las últimas restricciones son $2 * P$ (donde P es la cantidad de particiones), mientras que en Modelo Santiago son $N * P$. Tener más restricciones puede llevar a que el tiempo de cómputo sea mayor. %TODO La comparación de estos modelos está en la sección \ref{comparacion-modelos}

%TODO
La implementación de estos modelos se puede observar en la sección COMPLETAR. (Apendice o seccion del informe? Vale la pena poner codigo de esto en el informe?)

\section{Demostraciones}

En esta sección vamos a probar la validez de dos demostraciones que luego van a ser usadas para crear los distintos planos de corte.

\textbf{Desigualdad 1:} Sea $j_0 \in {1, \ldots, n}$ y sea $K$ una clique maximal de $G$. La desigualdad clique está definida por:
\begin{align*}
\sum_{p \in K} x_{pj_0} \leq w_{j_0}
\end{align*}

\textbf{Demostración:}
$w_{j_0} \in \{0,1\}$ por definición de los $w$.

Recordamos que $x_{ij} \in \{0,1\}$ para todo $i, j$.

$0 \leq \sum_{p \in K} x_{pj_0}$ ya que todos los x son mayores o iguales que $0$.

Por un lado hay un nodo coloreado por partición y, en particular, hay a lo sumo uno con el color $j_0$ por partición por la inecuación (1) de la sección \ref{modelo-pedro}. Luego, si los nodos de la clique pertenecen a una misma partición, están obligados a que a lo sumo uno tenga el color $j_0$.

Si los nodos de la clique pertenecen a distinta partición, son vecinos por estar en la clique y también están restringidos a que a lo sumo uno de ellos tenga el color $j_0$, por la inecuación (2) de la sección \ref{modelo-pedro}.

Luego, $\sum_{p \in K} x_{pj_0} \leq 1$ ya que a lo sumo hay un nodo con el color $j_0$ por clique.

Luego, $0 \leq \sum_{p \in K} x_{pj_0} \leq 1$. Como las x toman solo valores enteros, $\sum_{p \in K} x_{pj_0} \in \{0,1\}$.

Luego, el único caso donde puede fallar la desigualdad es que $\sum_{p \in K} x_{pj_0} = 1$ y $w_{j_0} = 0$. Vamos a ver por el absurdo que esto es imposible.

Por definición de los $w$, $w_{j_0} = 0$ si ningún vértice tiene el color $j_0$. Como $\sum_{p \in K} x_{pj_0} = 1$, significa que $\exists$ un vértice en la clique $K$ con el color $j_0$. Por lo tanto, $w_{j_0} = 1$. ABS!

Luego, vale la desigualdad 1.

\hfill \ensuremath{\Box}

\textbf{Desigualdad 2:} Sea $j_0 \in {1, \ldots, n}$ y $C_{2k+1} = v1, \ldots, v_{2k+1}$, $k \geq 2$, un agujero de longitud impar. La desigualdad odd-hole está definida por:
\begin{align*}
\sum_{p \in C_{2k+1}} x_{pj_0} \leq  k * w_{j_0}
\end{align*}

\textbf{Demostración:}

Al colorear un nodo con el color $j_0$, puedo verlo como remover las dos aristas que lo comunican con los otros dos nodos vecinos del agujero impar. En este sentido, si uno de los nodos no tiene dos aristas, significa que uno de sus vecinos está coloreado con el color $j_0$.

De esta forma, para colorear $k+1$ nodos con el color $j_0$ necesito $2 * (k+1)$ aristas. Sin embargo, el agujero impar tiene $2 * k$ aristas. Luego, no puedo colorear $k+1$ nodos con el color $j_0$.

Luego, $\sum_{p \in C_{2k+1}} x_{pj_0} \leq k$.

Por definición de los $w$, el $w_{j_0}$ puede ser 0 o 1.

Si $w_{j_0} = 0$, esto implica que ningún x está coloreado con el color $j_0$.

Luego, $\sum_{p \in C_{2k+1}} x_{pj_0} = 0 \leq 0 = k * w_{j_0}$.

Si $w_{j_0} = 0$, como sabemos que $\sum_{p \in C_{2k+1}} x_{pj_0} \leq k$ vale que $\sum_{p \in C_{2k+1}} x_{pj_0} \leq k * 1 = k * w_{j_0}$.
 
Luego, vale la desigualdad 2.

\hfill \ensuremath{\Box}

\section{Branch and Bound}

\emph{Branch and bound} (A partir de ahora BB) es un algoritmo que consiste en enumerar soluciones candidatas, interpretándolas como un árbol de soluciones. BB explora las ramas de este árbol pero antes de enumerar las soluciones candidatas de una rama, dicha rama es comparada con la mejor solución actual y es descartada en caso que no pueda superarla.

Por suerte, una vez que tenemos el problema creado en cplex, pedirle que corra BB es una sola línea: llamar a CPXmipopt con el entorno y el lp.

En este TP, resolvemos los MIP llamando a BB con variables binarias, y resolvemos los LP llamándolo con variables de tipo float.

\section{Cut and Branch}

\emph{Cut and branch} (A partir de ahora CB) es un algoritmo que resuelve la relajación de un MIP de forma similar a BB, con el agregado de que en algún subconjunto de los nodos del árbol explorado se realizan cálculos extra en función de la solución del nodo actual para determinar posibles planos de corte y agregarlos dinámicamente al modelo. En el TP, sólo hacemos esto en la raíz del árbol. Si alguna solución para la relajación del MIP original viola la Desigualdad 1 o 2, entonces podemos agregar una nueva restricción al modelo original con la esperanza de que más adelante, cuando pasemos a resolver el LP no relajado, debamos explorar un área de soluciones factibles menor, y así encuentre una solución óptima más rápidamente.

Si $S$ es el poliedro de búsqueda de la relajación de nuestro MIP, con este algoritmo lo que hacemos es intentar reducirlo lo máximo posible hasta alcanzar (idealmente) a $conv(S)$, que es el poliedro de búsqueda más pequeño posible. Vimos en clase que las desigualdades clique generan planos de corte de alta calidad, pues cortan a $S$ exactamente en las caras del poliedro $conv(S)$.

\subsection{Llevándolo a la implementación}

Para implementar el algoritmo de CB, elegimos experimentalmente tres valores: CANT\_CICLOS\_CB, CANT\_RESTR\_CLIQUE y CANT\_RESTR\_AGUJERO, que determinan respectivamente cuántas veces buscamos un grupo de planos de corte para agregar al LP, una cota para la cantidad de cliques a agregar en cada iteración y una cota para la cantidad de agujeros agregados en cada iteración.

Tanto dameClique como dameAgujero son heurísticas golosas, que utilizamos para buscar de forma lo más rápido posible cliques o agujeros en el grafo que, dada la solución actual, violen alguna de las dos desigualdades con las que trabajamos en el TP.\\

\begin{algorithm}[H]
	\SetKwInOut{Input}{input}
	\Input{modelo original relajado}
	\For{ CANT\_CICLOS\_CB }{
		sol $=$ resolver(modelo) \\
		\For{ cada color posible }{
			\For{$i$ in $0..$ CANT\_RESTR\_CLIQUE }{
				\eIf{$ i==0 $}{
					clique = dameClique(sol, modelo, color, nodoInicial = el de maximo peso en sol)
				}{
					clique = dameClique(sol, modelo, color, nodoInicial = nodo elegido al azar)
				}
				\If{clique todavia no fue agregada al modelo y clique no vacía}{
					agregarRestriccionClique(modelo, clique)
				}
			}
			\For{$j$ in $0..$ CANT\_RESTR\_AGUJERO}{
				agujero = dameAgujero(sol, modelo, color) \\
				\If{agujero todavia no fue agregado al modelo y agujero no vacío}{
					agregarRestriccionAgujero(modelo, agujero)
				}
			}
		}
	}	
\caption{Agrega restricciones de tipo clique y tipo agujero}
\end{algorithm}


\begin{algorithm}[H]
	\textbf{dameClique}\\
	\SetKwInOut{Input}{input}
	\Input{solución actual (sol), modelo original relajado, conjunto de nodos $S$ inicial (con un único nodo) que forman una clique}
	\While{todavía se puede agregar un nodo a la clique generada por $S$}{
		agregar a $S$ el nodo de mayor peso en $sol$ de aquellos nodos que pueden ser agregados a $S$ de forma
		tal que $S$ siga generando una clique
	}
	\eIf{la suma total de la clique es mayor a 1 + epsilonClique}{
		exito: devolver $S$
	}{
		fracaso: devolver una clique vacia
	}
\caption{dameClique genera una clique que viole la desigualdad 1}
\end{algorithm}

\begin{algorithm}[H]
	\textbf{dameAgujero}\\
	\SetKwInOut{Input}{input}
	\Input{solucion actual (sol), modelo original relajado, conjunto de nodos $S$ inicial (con un unico nodo) }
	\While{todavia se puede agregar un nodo al camino simple generado por $S$}{
		agregar a $S$ el nodo de mayor peso en $sol$ de aquellos nodos que pueden ser agregados a $S$ de forma
		tal que $S$ siga siendo un camino simple \\
		\eIf{se puede agregar un nodo $x$ a $S$ de forma tal que $S$ genere un ciclo impar y ademas el peso
		total de $S$ sea mayor a la mitad del tamaño del agujero}{
			exito: devolver $S$
		}{
			fracaso: devolver un agujero vacio
		}
	}
\caption{dameAgujero genera un agujero que viole la desigualdad 2}
\end{algorithm}


La función dameClique es una función determinística. Originalmente, la llamábamos con el nodo de mayor peso. Esto generaba que realizar más de un llamado a dameClique no tenga sentido ya que devolvía siempre la misma clique. Por esto, decidimos buscar una clique con el nodo más pesado y el resto de las iteraciones comenzar desde un nodo elegido completamente al azar. Esto provoca que encuentre más restricciones, y como nos aseguramos de hacer un llamado con el nodo más pesado no encontramos menos que antes.

%TODO grafico mostrando random vs no random?

La función dameAgujero también es determinística por lo que realizar sucesivos llamados no devuelve nuevos agujeros. El problema con realizar la misma idea que con clique es que con agujero el número a encontrar para violar depende de la cantidad de nodos del agujero (En contraposición, clique siempre compara contra 1) Entonces, comenzar con un nodo con poco peso implica que podemos no llegar a encontrar ningún agujero que viole la desigualdad. Por esta razón, decidimos dejar solamente el algoritmo determinístico.

Una vez que encontramos una clique o agujero violado, lo pasamos a agregar a nuestro problema lineal. Experimentando, vimos que sucedía algo similar a esto: comenzaban todos los nodos con los colores 1 y 2 con peso 0.5. Encontrabamos una clique violada y la agregábamos. Ahora los nodos usaban los colores 3 y 4 con 0.5, y así sucesivamente. Para evitar esta ``burla'' por parte de los nodos, una vez que encontramos una clique o agujero violado, lo agregamos para todos los colores posibles.

Para no estar agregando restricciones de más, guardamos las restricciones agregadas en cada iteración de planos de corte y comparamos la que vamos a agregar con las agregadas anteriormente. Como en la nueva iteración, no vamos a encontrar la misma violación estas ``restricciones agregadas'' pueden ser limpiadas en cada iteración de planos de corte y así liberar memoria.

\section{Desarrollo}

Al momento de experimentar, lo hicimos con dos recorridos del árbol (\emph{Depth-first search} y \emph{Best-bound search}) y con tres variables de corte (\emph{Minimum infeasibility}, \emph{CPLEX decide}, \emph{Maximum infeasibility}).

Para esta sección vamos a presentar los gráficos en los cuales vamos a comparar seis valores que se corresponden a las combinaciones de recorridos del árbol y las variables de corte.

Realizamos uno de estos gráficos para cada combinación de archivo (``david'', ``myciel'',  ``input0''), cada algoritmo (``BB'' y ``CB'') y cada modelo(``Modelo Pedro'' y ``Modelo Santiago''). A su vez, cada archivo representa un grafo que lo particionamos de distintas maneras(``notrandom'', ``random0.2'', ``random0.4'', ``random0.8'', ``random1.0'').

Esto genera una cantidad enorme de gráficos, y muchos de ellos no sumaban al informe y decidimos mostrar los más interesantes. Algunos de estos gráficos omitidos. se presentan en el Apéndice \ref{apendice-graficos} para mostrar algún que otro detalle.

%TODO 50 SEGUNDOS TOPE, habloar de las distintas semillas. Explicar que vamos a hablar de archivos, para luego pasar a comparaciones "longitudinales"

\subsection{Partición de grafos}

Para generarnos distintos grafos a partir de un mismo archivo, creamos un particionador de grafos.

Si elegimos la opcion ``notrandom'', nos va a generar una partición por nodo.

Si elegimos ``random'' y le damos una proporción (entre 0 y 1), nos va generar $proporcion * cantNodos$ particiones. A cada una de estas particiones le asignamos nodos al azar, y luego borramos las particiones vacías. De esta forma, tenemos como mínimo una partición y a lo sumo $proporcion * cantNodos$ particiones.

\subsection{Experimentación de ``input0''}

Este grafo presenta la particularidad de ser muy chico: 4 nodos y 4 aristas. Nuestra intención era ver si al usar un grafo lo suficientemente pequeño, convenía usar BB en vez de CB. En los grafos siguientes comparamos las distintas maneras de recorrer un árbol y las variables de corte con el Modelo Pedro, comparando CB vs BB.

\diagramavfig{input0_notrandom_1_bb_0_segunJuntada}{input0 notrandom usando Modelo Pedro y BB}
\diagramavfig{input0_notrandom_1_cb_0_segunJuntada}{input0 notrandom usando Modelo Pedro y CB}

Como se puede ver en los gráficos, BB efectivamente tarda menos que CB. Sin embargo, no rindió muchos frutos ya que los distintos resultados brindaban tiempos similares, en el rango de los pocos milisegundos. Estos milisegundos de diferencia se pueden asignar a varias razones, inclusive al \emph{scheduling}. Variar algoritmos, modelos y particiones no brindó cambios importantes.

\subsection{Experimentación de ``myciel''}

Luego, pasamos a experimentar con otro archivo: \emph{myciel}. Es un grafo con 11 nodos y 20 aristas. Similarmente que con \emph{input0}, \textit{myciel} tiene un tiempo comparable en CB y BB. Variar los distintos parámetros no puede hacer mejorar a CB lo suficiente como para lograr superar ampliamente a BB.

Es más, usando el Modelo Santiago BB logra ser más rápido que CB en el siguiente caso:

\diagramavfig{myciel3_notrandom_1_bb_1_segunJuntada}{myciel3 notrandom usando Modelo Santiago y BB}
\diagramavfig{myciel3_notrandom_1_cb_1_segunJuntada}{myciel3 notrandom usando Modelo Santiago y CB}

Salvo la peor elección de BB es peor que la mejor elección de CB. Cabe aclarar que esto no sucede en el Modelo Pedro, y este último se comporta de manera mejor o igual usando CB.

Al usar random la cantidad de particiones baja (ya que el tope es un color por nodo), y esto produce que tengamos que colorear unos pocos nodos. Por esto, \textit{myciel} tarda poco en correr (menos de un segundo) para las instancias que no son una partición por nodo.

\subsection{Experimentación de ``david''}

\textit{David} es nuestra instacia más grande: 87 nodos 406 aristas. Presenta un panorama parecido a las demás: más particiones, más tiempo de ejecución. Debido al tamaño de esta instancia, acá podemos ver claramente el beneficio de usar CB.

\diagramavfig{david_notrandom_1_bb_0_segunJuntada}{david notrandom usando Modelo Pedro y BB}
\diagramavfig{david_notrandom_1_cb_0_segunJuntada}{david notrandom usando Modelo Pedro y CB}

Recordamos que 50 segundos es el tope que corrimos por la instancia david, por lo que podría haber corrido todavía más tiempo. Por más que haya tardado 50 segundos, CB mejora 20 veces el tiempo de BB utilizando todas las diferentes maneras de recorrer el árbol y las distintas variables de corte.

En este mismo caso, el modelo Santiago presenta mejoras significativas (tardando alrededor de 10 segundos) pero el Modelo Pedro supera una vez más el tiempo de ejecución.

Con \emph{david}, BB no logra terminar en menos de 50 segundos para ``notrandom'', ``random0.8'', o ``random1'' que son las instancias con más particiones. Según la semilla, CB logra terminar en unos pocos segundos, o tarda más de 50 segunods en terminar.

\subsection{Comparaciones longitudinales}

\subsubsection{Comparación entre modelos}

El Modelo Pedro se comporta mejor que el Modelo Santiago en la mayoría de los casos. Incluso cuando se comporta peor, la diferencia es de unos pocos milisegundos. Observando detenidamente las salidas pudimos ver que la diferencia está en el tiempo que tarda en las iteraciones de planos de corte.

Modelo Pedro tarda un tiempo similar en CB respecto de BB, pero Modelo Santiago tarda aproximadamente diez veces más en esta parte del algoritmo. Curiosamente, ambos modelos terminan teniendo un tiempo similar al momento de realizar BB. De esta forma, Modelo Santiago tarda más al buscar los planos de corte pero termina llegando a un resultado similar que Modelo Pedro al momento de pasar a BB. Recordamos que estos resultados provienen de distintas semillas para asegurarnos que los resultados no son \textit{outliers}.

Cabe aclarar que cuando un modelo no puede encontrar un resultado en el tiempo dado, el otro modelo tampoco lo puede hacer. Es decir, si bien Modelo Pedro tarda menos tiempo en correr, no pudimos encontrar casos donde un modelo se comporte muchísimo mejor que otro.


\subsubsection{Proporcion de particiones}
Otro aspecto que nos pareció interesante analizar es la relación entre el tiempo de ejecución y la cantidad de particiones utilizada. A mayor cantidad de particiones, se deberá colorear una mayor cantidad de nodos, respetándose además las reglas de un coloreo particionado de grafos. Y, al tener más nodos coloreados, habrá que analizar una mayor cantidad de soluciones factibles y compararlas entre sí hasta encontrar una que sea la óptima. Por eso, es esperable que a mayor cantidad de particiones, mayor sea el tiempo de ejecución.

Para esto, generamos los siguientes gráficos para un mismo caso de test, variando únicamente el porcentaje de particiones utilizado (20\%, 40\% y 80\%). Utilizamos \emph{david}, que consta de 87 nodos y 406 aristas. Notar que los gráficos no están escalados de forma proporcional, y es importante observar los tiempos que figuran en los mismos para poder compararlos correctamente.

\diagramavfig{david_random_0.2_bb_0_segunJuntada}{david random 0.2 usando Modelo Pedro y BB}
\diagramavfig{david_random_0.4_bb_0_segunJuntada}{david random 0.4 usando Modelo Pedro y BB}
\diagramavfig{david_random_0.8_bb_0_segunJuntada}{david random 0.8 usando Modelo Pedro y BB}

Se observa que a mayor porcenaje de particiones, mayor es el tiempo que tarda nuestro algoritmo. En particular, ya con un 80 \% de particiones, el se sobrepasa nuestro límite de tiempo, que es de 50 segundos. Además, en los tres casos se puede notar que no hay diferencias significativas al variar la variable de corte o la forma de recorrer el árbol.

\subsubsection{Variables de corte y Recorrido del árbol}
En general, no observamos cambios importantes en 

\subsubsection{CB vs BB}
	
%TODO ver cuales agregamos

%nombreArchivo / algoritmo / modelo (esto puede cambiarse de orden)

%Dfs pareciera andar mejor
%Cuando hay muchas particioned
%en myciel se re ve eso
%ahi tambien se re ve la diferencia entre cb y bb
%y tambien esta bueno myciel para ver la diferencia entre 0.2 y 1 en % de particiones
%input0 no se para que lo podriamos usar
%para recorrido arbol tal vez



%\subsection{Comparación Modelos}
%\label{comparacion-modelos}

%\subsection{Problemas encontrados}

%\subsection{Conclusiones}




\newpage

\appendix

\section{Código Relevante}

En esta sección vamos a mostrar los fragmentos de código más importantes de nuestra implementación.

\subsection{Variables Globales}

Para facilitarnos el modelado, decidimos tener algunas variables globales:

\begin{lstlisting}
int N; // |Vertices|
int E; // |Aristas|
int P; // |Particiones|
vector <vector <int> > M; // Matriz de adyacencia. M_v1_v2 = 1 si (v1,v2) in E; 0 sino.
vector <vector <int> > S; // en S_p estan los vertices de la particion p.
\end{lstlisting}

\subsection{Ordenamiento de variables}

Las variables las guardamos de la siguiente forma:
\begin{center}
$w_0, w_1, \ldots, w_{(P-1)}, x_{00}, \ldots, x_{0(P-1)}, x_{10}, \ldots, x_{1(P-1)}, \ldots, x_{(N-1)0},\ldots, x_{(N-1)(P-1)}$
\end{center}

Es decir, primero los $w$ que representan los colores. Luego ordenamos por variable, y dentro de cada variable ordenamos por color.

Utilizamos la siguiente función para obtener el índice correcto, dado un nodo y un color:

\begin{lstlisting}
///Al principio me muevo en los indices del color por lo que lo muevo ese offset
///Para ir al indice correcto, me voy hasta cantColoresPorVariable * indiceVariable
///Luego, sumo el indice del color para moverme al color correcto dentro de la variable
int xijIndice(int indiceVariable, int indiceColor){
    return P + P * indiceVariable + indiceColor;
}
\end{lstlisting}


\subsection{Agregar Variables}

\begin{lstlisting}
// Definimos las variables. En total, son P + N*P variables ( las W[j] y las X[i][j] )
int cantVariables = P + N*P;
double *ub, *lb, *objfun; // Cota superior, cota inferior, coeficiente de la funcion objetivo.
char *xctype, **colnames; // tipo de la variable , string con el nombre de la variable.

for (int i = 0; i < cantVariables; i++) {
		ub[i] = 1.0; // seteo upper y lower bounds de cada variable
		lb[i] = 0.0;
		if(i < P) {  // agrego el costo en la funcion objetivo de cada variables
				objfun[i] = 1;  // busco minimizar Sum(W_j) para j=0..P (la cantidad de colores utilizados).
		} else {
				objfun[i] = 0;  // los X[i][j] no contribuyen a la funcion objetivo
		}
		xctype[i] = 'B';  // Binaria para Branch and Bound
		colnames[i] = new char[10];
}

for(int j=0; j<P; j++) {
		sprintf(colnames[j], "W_%d", j);
}
for(int i=1; i<=N; i++) {
		for(int j=0; j<P; j++) {
				sprintf(colnames[i*P+j], "X_%d_%d", i-1, j);
		}
}

if(algoritmo == "cb" ) {
	// si quiero resolver la relajacion, agregar los cortes y despues resolver el MIP, no agrego xctype
	CPXnewcols(env, lp, cantVariables, objfun, lb, ub, NULL, colnames);
} else if (algoritmo == "bb") {
	// si quiero hacer MIP, directamente, con branch and bound, agrego xctype
	CPXnewcols(env, lp, cantVariables, objfun, lb, ub, xctype, colnames);
}
\end{lstlisting}

\subsection{Agregar Restricciones}

Para las restricciones usamos las siguientes variables:

\begin{lstlisting}
double *rhs; // Termino independiente de las restricciones.
int *matbeg; //Posicion en la que comienza cada restriccion en matind y matval.
int *matind; // Array con los indices de las variables con coeficientes != 0 en la desigualdad.
double *matval; // Array que en la posicion i tiene coeficiente (!= 0) de la variable matind[i] en la restriccion.
\end{lstlisting}

Restricciones de un color por partición:

\begin{lstlisting}
// i) P restricciones - exactamente un color a cada vertice (una restriccion por cada particion)
for(int particion = 0; particion < P; particion++) {
	matbeg[cantRestricciones] = nzcnt;
	rhs[cantRestricciones]    = 1;
	sense[cantRestricciones]  = 'E';
	for(int e = 0; e < S[particion].size(); e++) {
		for(int color = 0; color < P; color++) {
			matind[nzcnt] = xijIndice(S[particion][e], color);
			matval[nzcnt] = 1;
			nzcnt++;
		}
	}
			cantRestricciones++;
}
\end{lstlisting}

Restricciones de vecinos con distinto color:

\begin{lstlisting}
// ii) Cota superior de (E*P)/2 restricciones mas
// Una para cada par de vecinos i j, para cada color pero solo cuando i < j, y estan en distinta particion
for(int i = 0; i < N; i++) {
	for(int j = i + 1; j < N; j++) { 
		if(M[i][j] == 1 and dameParticion(i) != dameParticion(j)){
			for(int color = 0; color < P; color++) {
				matbeg[cantRestricciones] = nzcnt;
				rhs[cantRestricciones]    = 1;
				sense[cantRestricciones]  = 'L';

				matind[nzcnt] = xijIndice(i,color);
				matval[nzcnt] = 1;
				nzcnt++;
				matind[nzcnt] = xijIndice(j,color);
				matval[nzcnt] = 1;
				nzcnt++;
				cantRestricciones++;
			}
		}
	}
}
\end{lstlisting}

Y acá es donde se diferencian los modelos: en cómo modelar la relación entre los nodos y sus colores asociados.

\subsubsection{Modelo Pedro}
\label{codigo-modelo-pedro}

\begin{lstlisting}
// iii) 2*P restricciones mas
// - P * wj + sigma xij <= 0
for(int k=0; k<P; k++) {  // para cada color
		matbeg[cantRestricciones] = nzcnt;
		rhs[cantRestricciones] = 0;
		sense[cantRestricciones] = 'L';
		matind[nzcnt] = k;
		matval[nzcnt] = -1 * P;
		nzcnt++;
		for(int i=0; i<N; i++) {
				matind[nzcnt] = xijIndice(i,k);
				matval[nzcnt] = 1;
				nzcnt++;
		}
		cantRestricciones++;
}

//  - wj + sigma xij >= 0
for(int k=0; k<P; k++) {
		matbeg[cantRestricciones] = nzcnt;
		rhs[cantRestricciones] = 0;
		sense[cantRestricciones] = 'G';
		matind[nzcnt] = k;
		matval[nzcnt] = -1;
		nzcnt++;
		for(int i=0; i<N; i++) {
				matind[nzcnt] = xijIndice(i,k);
				matval[nzcnt] = 1;
				nzcnt++;
		}
		cantRestricciones++;
}
\end{lstlisting}

\subsubsection{Modelo Santiago}
\label{codigo-modelo-santiago}

\begin{lstlisting}
// iii) N*P restricciones mas
// -wj + xij <= 0
for(int color = 0; color < P; color++) { 
		for(int i = 0; i < N; i++) {
				matbeg[cantRestricciones] = nzcnt;
				rhs[cantRestricciones] = 0;
				sense[cantRestricciones] = 'L';
				matind[nzcnt] = color;
				matval[nzcnt] = -1;
				nzcnt++;
				matind[nzcnt] = xijIndice(i, color);
				matval[nzcnt] = 1;
				nzcnt++;
				cantRestricciones++;
		}
}
\end{lstlisting}


\newpage

\section{Gráficos extra}
\label{apendice-graficos}


\newpage

\section{Cómo correrlo}
\label{como-correrlo}






\end{document}
